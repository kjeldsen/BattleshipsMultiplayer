\documentclass[a4paper,11pt]{article}

% Use utf-8 encoding for foreign characters
\usepackage[utf8]{inputenc}
\usepackage[british,english]{babel}
\usepackage[T1]{fontenc}

% Setup for fullpage use
\usepackage{fullpage}

% Multipart figures
\usepackage{subfigure}

% More symbols
\usepackage{amsmath}
\usepackage{amssymb}
\usepackage{latexsym}

% Surround parts of graphics with box
\usepackage{boxedminipage}

% Package for including code in the document
\usepackage{listings}

% If you want to generate a toc for each chapter (use with book)
\usepackage{minitoc}

% Uncomment if you want to use Palatino as font
\usepackage[sc]{mathpazo}
\linespread{1.05}         % Palatino needs more leading (space between lines)

% This is now the recommended way for checking for PDFLaTeX:
\usepackage{ifpdf}

%\newif\ifpdf
%\ifx\pdfoutput\undefined
%\pdffalse % we are not running PDFLaTeX
%\else
%\pdfoutput=1 % we are running PDFLaTeX
%\pdftrue
%\fi

\ifpdf
\usepackage[pdftex]{graphicx}
\else
\usepackage{graphicx}
\fi
\title{Product and Process Brief\\\small{for}\\\small{Danske Bank: Peer-to-peer}}
\author{ Jesper Borgstrup, Thomas Kjeldsen and Mads Ohm Larsen }

\date{February 18}

\begin{document}

\ifpdf
\DeclareGraphicsExtensions{.pdf, .jpg, .tif}
\else
\DeclareGraphicsExtensions{.eps, .jpg}
\fi

\maketitle

\tableofcontents
\vspace{2cm}

\noindent\textbf{Terms}

The term ``task'' is often used in the team charter. 
It substitutes all RTC element types – e.g. business deliverable, defect etc. unless otherwise stated.

\pagebreak
Multiple reasons exist for preparing a team charter. One is to document the team's purpose and clearly define individual roles, responsibilities, and operating rules. Next, it establishes procedures for both the team and agency management on communicating, reporting, and decision-making procedures. It lays out a blueprint for conducting business for the acquisition and defines how the team works in an empowered manner, including setting out responsibility and authority. Finally it facilitates stakeholder buy in by including key members in the decision making process and obtaining their concurrence along the way. 

\section{Purpose/Vision} % (fold)
\label{sec:purpose_vision}

The aim of this project is to provide an API, with which the Danske Bank (DB) Android App could be extended. The proposed extension will facilitate easy transfer of money from one DB customer's account to another DB customer's account, by merely bumping their phones together.

This extension of the app aligns with the company's marketing strategy, by adding another ``cool'' technology to the app, which users will hopefully show off to friends and family similar to the recent/current voice recognition ``craze'' (Voice Search, Voice and Translate from Google, Tellme voice search for Windows Phone 7 etc.).

In addition to the social advertising aspects it further solidifies Danske Banks presence in the mobile area, which is one the factors currently differentiating DB from it's competitors for Danish private customers. 

Finally it's worth noting that PayPal, which is a major international player in micro-payments, currently has Bump functionality in their mobile app. While PayPal does not pose a significant threat to DB at present time it's worth staying on the forefront in case of a sudden surge in either micro-payments or Bump-technology ``craze''.

\subsection{Roadmap} % (fold)
\label{subsec:roadmap}
Our roadmap is divided into our five sprints.

These five sprints, and their goals, are as follows:
\begin{itemize}
	\item Sprint \#1 (March 11$^{\text{th}}$, 2011)
	\begin{itemize}
		\item Research bump technology, provide working prototype of bump between two Android smartphones
	\end{itemize}
	
	\item Sprint \#2 (April 1$^{\text{st}}$, 2011)
	\begin{itemize}
		\item Making safe transfers between two phones
	\end{itemize}
	
	\item Sprint \#3 (April 29$^{\text{th}}$, 2011)
	\begin{itemize}
		\item UI, polishing, user experience tested
	\end{itemize}
	
	\item Sprint \#4 (May 20$^{\text{th}}$, 2011)
	\begin{itemize}
		\item Undecided (kept open, due to project uncertainty)
	\end{itemize}
	
	\item Sprint \#5 (June 10$^{\text{th}}$, 2011)
	\begin{itemize}
		\item Final product delivered to customer
		\item Paper finished and submitted
	\end{itemize}
\end{itemize}
% subsection roadmap (end)

% section purpose_vision (end)

\section{User/Customer} % (fold)
\label{sec:user_customer}

Our customer is Danske Bank and Sune Lomholt has agreed to be out product owner.

% section user_customer (end)

\section{Team resources, roles and obligations} % (fold)
\label{sec:team_resources_roles_and_obligations}
The team consists of three Master's students at the Department of Computer Science of the University of Copenhagen (DIKU): \\

\begin{tabular}{|p{4.5cm}|p{5cm}|p{3.5cm}|}
\hline
\textbf{Name}    & E-mail				          &	Telephone         \\\hline
Jesper Borgstrup & {\tt jesper@borgstrup.dk} 	  & (+45) 61 30 30 81 \\\hline
Thomas Kjeldsen  & {\tt thomas@thomaskjeldsen.dk} & (+45) 61 30 80 01 \\\hline
Mads Ohm Larsen  & {\tt mads.ohm@gmail.com} 	  & (+45) 60 16 39 53 \\\hline
\end{tabular}

\subsection{Skills} % (fold)
\label{subsec:skills}
In our team we will need some different skills.
This is the distribution on how we find ourselves fitted for these skills: \\

\begin{tabular}{|p{4.5cm}|p{3cm}|p{3cm}|p{3cm}|}
\hline
\textbf{Skills}          & \textbf{High}   & \textbf{Medium} & \textbf{Low} 					 \\\hline
Android development      & Jesper & Thomas & Mads 					 \\\hline
Configuration Management & Mads & Jesper and Thomas          &      					 \\\hline
Scrum Master             &        &        & Jesper, Thomas and Mads \\\hline 
User involvement         &        & Mads and Jesper & Thomas \\\hline
Test                     & Mads and Thomas   & Jesper & \\\hline
\end{tabular}
% subsection skills (end)

\subsection{Resources} % (fold)
\label{subsec:resources}
Correspondence between team members and roles:\\

\begin{tabular}{|p{3cm}|p{5cm}|p{5cm}|}
\hline
\textbf{Team Member} & \textbf{Roles and responsibilities}   & \textbf{Strengths and focus area} \\\hline
Jesper               & Coder                   & Development \\\hline
Thomas               & Scrum Master, tester    & Test                 \\\hline
Mads                 & SCM maintainer, tester  & Test, user involvement                         \\\hline 
\end{tabular}
\\
\\
These roles are very loose, and will be subject to change throughout the course. E.g. we will all be Scrum Master at some point during this course.
% subsection resources (end)

% section team_resources_roles_and_obligations (end)

\section{Team empowerment} % (fold)
\label{sec:team_empowerment}
\begin{itemize}
	\item The team breaks down and estimates prioritised work items in the sprint backlog

	\item The team jointly determines how to perform the work, including possible ad hoc planning meetings

	\item Each team member plans their own daily work in respect for planned activities/meeting
\end{itemize}

% section team_empowerment (end)

\section{Team values} % (fold)
\label{sec:team_values}
The following are agreements made between the project members

\begin{itemize}
	\item We're respectfully nice to each other.

  	\item If someone's feeling overwhelmed they're entitled to a hug (subject to a daily maximum of 3 and a weekly maximum of 5)

  	\item We show up on time, and if we're prevented or running late we let each other know.

  	\item We speak openly about problems

  	\item We share information, be that information from or to product owner, or between us, via e-mail, the configuration management system or other ways

  	\item We participate equally, meaning that everybody have an equal work load
\end{itemize}

% section team_values (end)

\section{Team processes} % (fold)
\label{sec:team_processes}

\begin{itemize}
	\item Our code will be open for everybody to see
	
	\item We have agreed on using git, and our code will be stationed at GitHub
	
	\item We have agreed on working together in person, but also being available on both a messenger platform and on Skype - or eMeeting
	\begin{itemize}
		\item eMeeting is a platform suggested by Sune Lomholt
	\end{itemize}
\end{itemize}

\subsection{Configuration management} % (fold)
\label{subsec:configuration_management}

We use Git as our SCM system. 
Our repository is hosted publicly on GitHub.

We're using a Test Driven Development approach, which imposes the following structure on the \texttt{master} branch:
\begin{itemize}
\item Code committed to the central repository must be thoroughly covered by JUnit tests.
\item Tests must not fail when comitted, even if this means methods are stubbed.
\item Code must compile.
\end{itemize}

% subsection configuration_management (end)

\subsection{Office rules} % (fold)
\label{subsec:office_rules}
When working together, e.g. sitting together and working on stories, the following should be respected:

\begin{itemize}
	\item No loud music or noises

	\item Procrastination should not affect others
\end{itemize}
% subsection office_rules (end)

\subsection{Calendar planning} % (fold)
\label{subsec:calendar_planning}

We have agreed to do a daily virtual scrum logging, where each participant writes an entry firstly about what he did today, what he will do tomorrow, and if any problems have occurred.

Sprint demos will occur at, or around, the ending of a sprint, depending on when the product owner have got time.
At the time of sprint demo, we will properly want to do a sprint planning, for the following sprint, together with the product owner.

At the end of a sprint, we will have a so-called sprint retrospective, discussing what went well and what could be approved for next sprint.

% subsection calendar_planning (end)

% section team_processes (end)

\section{Team performance \& progress monitoring} % (fold)
\label{sec:team_performance_progress_monitoring}
We will use Google Docs as our scrum tool, but we're planning to (briefly) evaluate some of the scrum tools mentioned at lectures, and also IBM's JAZZ.

The purpose of a scrum tool is to be able to have a virtual storyboard and burndown.
% section team_performance_progress_monitoring (end)


\end{document}
