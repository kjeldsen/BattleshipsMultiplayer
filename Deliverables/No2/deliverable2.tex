\documentclass[a4paper,11pt]{article}

% Use utf-8 encoding for foreign characters
\usepackage[utf8]{inputenc}
\usepackage[british,english]{babel}
\usepackage[T1]{fontenc}

% Setup for fullpage use
\usepackage{fullpage}

% Multipart figures
\usepackage{subfigure}

% More symbols
\usepackage{amsmath}
\usepackage{amssymb}
\usepackage{latexsym}

% Surround parts of graphics with box
\usepackage{boxedminipage}

% Package for including code in the document
\usepackage{listings}

% If you want to generate a toc for each chapter (use with book)
\usepackage{minitoc}

% Uncomment if you want to use Palatino as font
\usepackage[sc]{mathpazo}
\linespread{1.05}         % Palatino needs more leading (space between lines)

% This is now the recommended way for checking for PDFLaTeX:
\usepackage{ifpdf}

%\newif\ifpdf
%\ifx\pdfoutput\undefined
%\pdffalse % we are not running PDFLaTeX
%\else
%\pdfoutput=1 % we are running PDFLaTeX
%\pdftrue
%\fi

\ifpdf
\usepackage[pdftex]{graphicx}
\else
\usepackage{graphicx}
\fi
\title{Product and Process Brief\\\small{for}\\\small{Danske Bank: Peer-to-peer}}
\author{ Jesper Borgstrup, Thomas Kjeldsen and Mads Ohm Larsen }

\date{February 18}

\begin{document}

\ifpdf
\DeclareGraphicsExtensions{.pdf, .jpg, .tif}
\else
\DeclareGraphicsExtensions{.eps, .jpg}
\fi

\maketitle

\tableofcontents
\vspace{2cm}

\section{temporary notes}

migrer til accunote

Klaus:
* ideen
* kan vi køre DB ud på et sidespor
* final product: library, sample app



powerconsumption per kb
* bluetooth
* gprs/3g

value proposition
define sprints:
1 (2 uger): bump, bluetooth MAC + application UUID, secret key + researchet + testet
2: bluetooth forbindelse (sikker), UI polished
3: library, vores API
4: konkret sample application showcasing either paypal/DB transfers, chat or filetransfers...
5: wrap up

læst op på bluetooth API
Diffie-Helman cryptography: http://developer.android.com/reference/javax/crypto/interfaces/package-summary.html
undersøg nok med bluetooth encryption?
bump overførsel af ???: secretkey + bluetooth UUID


3:

bump pairing OK
bluetooth discovery på 1 (og evt. bluetooth on på 2)
bluetooth pairing på begge (første gang)


Hvis getRemoteDevice virker, så slipper vi for discovery:

bump pairing OK
(evt. bluetooth on)
bluetooth pairing på begge (første gang)


tændt bluetooth
visibility on
beslut en eller begge telefoner: server (anden er bluetooth klient) <- overfør m. bump?
forbind / opsæt RFCOMM
krypteret kanalen

???

offentliggjort / tilgængelighed for andre: API
* opret
* luk
* send
* lyt

???

demo app muligheder:
* paypal
* sikker chat
* filoverførsel
(ikke søbet ind i vores SecureBTtransfers kode, men importerer library)

???

profit / world domination


optional/extensions/develop with this in mind:
other identity providers: NFC
XXX for root users

References:
> http://www.eetimes.com/design/communications-design/4012606/How-NFC-can-to-speed-Bluetooth-transactions-151-today
The two gamers only need to touch their devices and allow the NFC software on the devices to handle the discovery, inquiry, authentication and encryption set-up required to establish a Bluetooth link.\\
> http://developer.android.com/sdk/android-2.3.3.html
Bluetooth: Android 2.3.3 adds platform and API support for Bluetooth nonsecure socket connections. This lets applications communicate with simple devices that may not offer a UI for authentication. See createInsecureRfcommSocketToServiceRecord(java.util.UUID) and listenUsingInsecureRfcommWithServiceRecord(java.lang.String, java.util.UUID) for more information.\\
> http://mobisocial.stanford.edu/news/2011/02/nfc-and-bluetooth-brought-together-with-android-2-3-3/
To use NFC to kick off a Bluetooth session, first start a running socket server on one device, using a randomly generated service UUID. Then, create an NDEF message encoding the MAC address of the phone’s Bluetooth device and the UUID of the listening service. When this message is received by the joining phone, it will have enough information to connect to the Bluetooth server socket



% end temporary notes

\pagebreak 

\section{Purpose/Vision} % (fold)
\label{sec:purpose_vision}

The aim of this project is to provide a library, which Android App developers can utilize to facilitate easy secure communication between two phones over Bluetooth.

Providing secure communication can be reduced to two discting problems: 1) securing communication against eavesdropping and 2) validating the identity of the other party to protect against man-in-the-middle style attacks.

We will use Bluetooth as the transport medium over which the transfers take place, which in addition to cryptographic measures adds additional resilience by forcing a potential attacker to be in close proximity to the user.

Bluetooth offers advantages compared to transmitting data over the mobile network, since it is is quite efficient in terms of both power drain and transfer speed and it is not subject to usage fees.
*** SOMETHING ABOUT EASE (or rather lack thereof) of setting up Bluetooth connections (from a user perspective) ***

The library will be able to support multiple identity providers, which provide identification of the client and transport the neccessary information (out of band) for setting up the secure bluetooth transport. Support for using Bump as an identity provider will be included as a part of the project, but the library could be extended with further identity providers such as NFC (Near Field Communication) or 3rd party solutions e.g. a custom authentification server set up by a Bank or by NemID.

Information exchanged through Bump is subject to the Bump privacy policy, which specifically states that Bump is allowed to retain all exchanged information. For co-op games and other non-sensitive applications this may be good enough, but for exhange of financial information it is not neccesarily good enough. *** REWRITE/REMOVE: not true ***

*** REWRITE / REFOCUS below ***

We have chosen to focus our sample application and prototype to target financial institututions and bank where high security and transactional confidentiality is a prerequisite, but the library will be of value to anyone who wants to utilize the Bump API while safeguarding transactions.

Banks who use our library will be able to facilitate easy transfer of money from one customer's account to another customer's account, by merely bumping their phones together.

Danske Bank is one potential customer, since its marketing strategy is throughly based on a strong presence in both social media and on the mobile platform. Their mobile banking application already has a number of technology features and it is one of the key differentiating features compared to their competitors among Danish banks.

In addition to easing transactions, adding Bump would add yet another ``cool'' feature, which users will hopefully show off to friends and family similar to the recent/current voice recognition ``craze'' (Voice Search, Voice and Translate from Google, Tellme voice search for Windows Phone 7 etc.), thus adding further value through viral marketing.

Finally it's worth noting that PayPal, which is a major international player in micro-payments, currently has Bump functionality in their mobile app. While PayPal does not pose a significant threat to established financial institutions at present time it's worth staying on the forefront in case of a sudden surge in either micro-payments or Bump-technology ``craze''.

% TODO: Evt. egen bump-service?

\subsection{Roadmap} % (fold)
\label{subsec:roadmap}
Our roadmap is divided into our five sprints.

These five sprints, and their goals, are as follows:
\begin{itemize}
	\item Sprint \#1 (March 11$^{\text{th}}$, 2011)
	\begin{itemize}
		\item Read up on relevant documentation, APIs and product and process brief refined. Implementation of a fully working prototype that allows two Android smartphones to be bumped together and transfer information between them.
	\end{itemize}
\pagebreak
	\item Sprint \#2 (April 1$^{\text{st}}$, 2011)
	\begin{itemize}
		\item Sample App that can perform secure transfers between two phones. Functionality packaged as a distributable library.
	\end{itemize}
	
	\item Sprint \#3 (April 29$^{\text{th}}$, 2011)
	\begin{itemize}
		\item Different approaches as how to secure information in transit have been evaluated and benefits and drawbacks have been thoroughly examined. Prototype app can perform secure communication using the preferred approach.
	\end{itemize}
	
	\item Sprint \#4 (May 20$^{\text{th}}$, 2011)
	\begin{itemize}
		\item UI polished and user experience evaluated.
	\end{itemize}
	
	\item Sprint \#5 (June 10$^{\text{th}}$, 2011)
	\begin{itemize}
		\item Final product delivered to customer
		\item Paper finished and submitted
	\end{itemize}
\end{itemize}
% subsection roadmap (end)

% section purpose_vision (end)

\section{User/Customer} % (fold)
\label{sec:user_customer}

Our customers are Android app developers, banks and financial institutions in particular, who are interested in utilizing the Bump to facilitate communication or transactions between two phones, but are not interested in leaving entirely in the hands of the creators of Bump and having clear text data be subject to the Bump privacy policy.

Sune Lomholt has agreed to act as a representative of our product owner(s).

% section user_customer (end)

\section{Team resources, roles and obligations} % (fold)
\label{sec:team_resources_roles_and_obligations}
The team consists of three Master's students at the Department of Computer Science of the University of Copenhagen (DIKU): \\

\begin{tabular}{|p{4.5cm}|p{5cm}|p{3.5cm}|}
\hline
\textbf{Name}    & E-mail				          &	Telephone         \\\hline
Jesper Borgstrup & {\tt jesper@borgstrup.dk} 	  & (+45) 61 30 30 81 \\\hline
Thomas Kjeldsen  & {\tt thomas@thomaskjeldsen.dk} & (+45) 61 30 80 01 \\\hline
Mads Ohm Larsen  & {\tt mads.ohm@gmail.com} 	  & (+45) 60 16 39 53 \\\hline
\end{tabular}

\subsection{Skills} % (fold)
\label{subsec:skills}
In our team we will need some different skills.
This is the distribution on how we find ourselves fitted for these skills: \\

\begin{tabular}{|p{4.5cm}|p{3cm}|p{3cm}|p{3cm}|}
\hline
\textbf{Skills}          & \textbf{High}   & \textbf{Medium} & \textbf{Low} 					 \\\hline
Android development      & Jesper & Thomas & Mads 					 \\\hline
Configuration Management & Mads & Jesper and Thomas          &      					 \\\hline
Scrum Master             &        &        & Jesper, Thomas and Mads \\\hline 
User involvement         &        & Mads and Jesper & Thomas \\\hline
Test                     & Mads and Thomas   & Jesper & \\\hline
\end{tabular}
% subsection skills (end)

\subsection{Resources} % (fold)
\label{subsec:resources}
Correspondence between team members and roles:\\

\begin{tabular}{|p{3cm}|p{5cm}|p{5cm}|}
\hline
\textbf{Team Member} & \textbf{Roles and responsibilities}   & \textbf{Strengths and focus area} \\\hline
Jesper               & Coder                   & Development \\\hline
Thomas               & Scrum Master, tester    & Test                 \\\hline
Mads                 & SCM maintainer, tester  & Test, user involvement                         \\\hline 
\end{tabular}
\\
% TODO: Overvej ovenstående
\\
These roles are very loose, and will be subject to change throughout the course. E.g. we will all be Scrum Master at some point during this course.
% subsection resources (end)

% section team_resources_roles_and_obligations (end)

\section{Team empowerment} % (fold)
\label{sec:team_empowerment}
\begin{itemize}
	\item The team breaks down and estimates prioritised work items in the sprint backlog

	\item The team jointly determines how to perform the work, including possible ad hoc planning meetings

	\item Each team member plans their own daily work in respect for planned activities/meeting
\end{itemize}

% section team_empowerment (end)

\section{Team values} % (fold)
\label{sec:team_values}
The following are agreements made between the project members

\begin{itemize}
	\item We're respectfully nice to each other.

  	\item If someone's feeling overwhelmed they're entitled to a hug (subject to a daily maximum of 3 and a weekly maximum of 5)

  	\item We show up on time, and if we're prevented or running late we let each other know.

  	\item We speak openly about problems

  	\item We share information, be that information from or to product owner, or between us, via e-mail, the configuration management system or other ways

  	\item We participate equally, meaning that everybody have an equal work load
\end{itemize}

% section team_values (end)

\section{Team processes} % (fold)
\label{sec:team_processes}

\begin{itemize}
	\item Our code will be open for everybody to see
	
	\item We have agreed on using git, and our code will be stationed at GitHub
	
	\item We have agreed on working together in person, but also being available on both a messenger platform and on Skype - or eMeeting
	\begin{itemize}
		\item eMeeting is a platform suggested by Sune Lomholt
	\end{itemize}
\end{itemize}

\subsection{Configuration management} % (fold)
\label{subsec:configuration_management}

We use Git as our SCM system. 
Our repository is hosted publicly on GitHub.

We're using a Test Driven Development approach, which imposes the following structure on the \texttt{master} branch:
\begin{itemize}
\item Code committed to the central repository must be thoroughly covered by JUnit tests.
\item Tests must not fail when comitted, even if this means methods are stubbed.
\item Code must compile.
\end{itemize}

% subsection configuration_management (end)

\subsection{Office rules} % (fold)
\label{subsec:office_rules}
When working together, e.g. sitting together and working on stories, the following should be respected:

\begin{itemize}
	\item No loud music or noises

	\item Procrastination should not affect others
\end{itemize}
% subsection office_rules (end)

\subsection{Calendar planning} % (fold)
\label{subsec:calendar_planning}

We have agreed to do a daily virtual scrum logging, where each participant writes an entry firstly about what he did today, what he will do tomorrow, and if any problems have occurred.

Sprint demos will occur at, or around, the ending of a sprint, depending on when the product owner have got time.
At the time of sprint demo, we will properly want to do a sprint planning, for the following sprint, together with the product owner.

At the end of a sprint, we will have a so-called sprint retrospective, discussing what went well and what could be approved for next sprint.

% subsection calendar_planning (end)

% section team_processes (end)

\section{Team performance \& progress monitoring} % (fold)
\label{sec:team_performance_progress_monitoring}
To support our process we are interested in a scrum tool that, especially because we have no offices or static whiteboards, facilitates a virtual collaboration using the scrum model.

We are using Google Docs as our scrum tool and have set up a logbook of (virtual) daily scrum meetings, burndown chart, sprint- and product backlogs.

Throughout Sprint 1 we will will assess how the tools are working to ensure that they are being used to their maximum potential and they are providing the neccesary product support.

Pending this assessment we will consider evaluating additional scrum tools, such as IBM's Rational Team Concert (based on the Jazz collaboration platform).

% TODO: Der skal skrives at vi bruger Acunote?

% section team_performance_progress_monitoring (end)


\end{document}
