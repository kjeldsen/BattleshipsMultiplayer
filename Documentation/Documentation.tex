\documentclass[a4paper,11pt]{article}
\usepackage[utf8]{inputenc}
\usepackage[english]{babel}
\usepackage[usenames,dvipsnames]{color}
\usepackage{verbatim}
\usepackage{graphicx}
\usepackage{amsmath}
\usepackage{amssymb}
\usepackage{subfig}
\usepackage{ctable}
\usepackage{hyperref}

\newcommand{\term}[1]{{\bf #1}}

%\author{Jesper Borgstrup}
\title{DustyTuba Bluetooth Library Documentation}
%\date{Torsdag d. 7. april 2011}

\begin{document}
%\setcounter{page}{0}
\maketitle

\begin{abstract}
This document outlines the three basic steps needed to incorporate the DustyTuba Bluetooth Library in your Android application
\end{abstract}

%\tableofcontents

%\clearpage

%%%%%%%%%%%%%%%%%%%%%%%%%%%%%%%%%%%%%%%%%%
%%%%%%%%%%%%%%%%%%%%%%%%%%%%%%%%%%%%%%%%%%
%%%%%%%%%%%%%%%%%%%%%%%%%%%%%%%%%%%%%%%%%%

\section{Step one - Include necessary files}
Firstly, you need to copy the three folders (\verb+libs+, \verb+res+ and \verb+src+)-folder included in the archive to the root folder of the Android project in which you want to use the library.

Secondly, you must include the libraries in the \verb+libs+ folder into your project. In Eclipse, this can be done by right-clicking your project and selecting ``Properties'', then selecting ``Java Build Path'', and in the tab ``Libraries'' clicking the ``Add JARs...'' button and selecting the two included {\tt .jar} files in the \verb+libs+ folder.

As is also described in the Bump\texttrademark{} documentation, you must import your project's \verb+R+ class in the {\tt com.bumptech.bumpapi.BumpResources} source file -- i.e. if your project has package name {\tt com.example.test}, you must insert the following line into the {\tt BumpResources.java} file:

\begin{verbatim}
import com.example.test.R;
\end{verbatim}

\clearpage

%%%%%%%%%%%%%%%%%%%%%%%%%%%%%%%%%%%%%%%%%%
%%%%%%%%%%%%%%%%%%%%%%%%%%%%%%%%%%%%%%%%%%
%%%%%%%%%%%%%%%%%%%%%%%%%%%%%%%%%%%%%%%%%%

\section{Step two - Modify manifest}
Now you must expand the Android manifest to include the activities provided and to use the permissions needed. The activities you must declare goes in the \verb+<application>+-element and are the following:

\footnotesize
\begin{verbatim}
<!-- Reserved for future use: May replace all other activities when fully implemented -->
<activity android:name="dk.hotmovinglobster.dustytuba.id.GenericIPActivity" />
<!-- Mandatory for API -->
<activity android:name="dk.hotmovinglobster.dustytuba.bt.BluetoothConnector"></activity>
<!-- Optional: Identity Providers, include one or more -->
<activity android:name="dk.hotmovinglobster.dustytuba.id.FakeIPActivity" />
<activity android:name="dk.hotmovinglobster.dustytuba.id.ManualIPActivity" />
<activity android:name="dk.hotmovinglobster.dustytuba.id.BumpIPActivity"
    android:configChanges="keyboardHidden|orientation"/>
<!-- Optional: Additional activities required by BumpIPActivity -->
<activity android:name="com.bumptech.bumpapi.BumpAPI"
    android:configChanges="keyboardHidden|orientation"
    android:theme="@style/BumpDialog" />
<activity android:name="com.bumptech.bumpapi.EditTextActivity"
    android:configChanges="keyboardHidden|orientation"
    android:theme="@style/BumpDialog" />
\end{verbatim}
\normalsize

The permissions to be declared goes in the main \verb+<manifest>+-element and are the following:

\footnotesize
\begin{verbatim}
<!-- Mandatory for API -->
<uses-permission android:name="android.permission.BLUETOOTH" />
<!-- Optional: BLUETOOTH_ADMIN needed for cancel discovery (powersave) -->
<uses-permission android:name="android.permission.BLUETOOTH_ADMIN" />
<!-- Optional: Additional permissions required by BumpIPActivity -->
<uses-permission android:name="android.permission.ACCESS_FINE_LOCATION" />
<uses-permission android:name="android.permission.ACCESS_COARSE_LOCATION" />
<uses-permission android:name="android.permission.INTERNET" />
<uses-permission android:name="android.permission.VIBRATE" />
<uses-permission android:name="android.permission.READ_PHONE_STATE" />
\end{verbatim}
\normalsize

See appendix \ref{sample-manifest} for a complete sample manifest file.

\clearpage

%%%%%%%%%%%%%%%%%%%%%%%%%%%%%%%%%%%%%%%%%%
%%%%%%%%%%%%%%%%%%%%%%%%%%%%%%%%%%%%%%%%%%
%%%%%%%%%%%%%%%%%%%%%%%%%%%%%%%%%%%%%%%%%%

\section{Step three - Call the library}
\label{stepthree}
In order to call the library, you will use the \verb+BtAPI.getIntent()+ method (see section \ref{getIntent}) to generate an \verb+Intent+ which must be passed on to Android's \verb+startActivityForResult()+ method.

When the activity contained in the \verb+Intent+ finishes, you must override the \verb+onActivityResult()+ method of your main activity to handle the result and extract the Bluetooth connection object.

To receive data from the other end of the Bluetooth connection, you need to have a class implement the \verb+BtAPIListener+ interface (usually this class would be your main activity).

For example, to invoke the manual identity provider\footnote{Se section \ref{identity-providers} for a description of an identity provider}, insert the following code where you want it to be invoked:

\footnotesize
\begin{verbatim}
Intent i = BtAPI.getIntent(MainActivity.this, BtAPI.IDENTITY_PROVIDER_MANUAL);
startActivityForResult(i, REQUEST_DUSTYTUBA);
\end{verbatim}
\normalsize

where \verb+REQUEST_DUSTYTUBA+ is an integer constant chosen to distinguish between the result of this activity and others you may be using. Also make your main activity implement the \verb+BtAPIListener+ interface and override the \verb+onActivityResult()+ method as follows:

\footnotesize
\begin{verbatim}
@Override
protected void onActivityResult (int requestCode, int resultCode, Intent data) {
  if (requestCode == REQUEST_DUSTYTUBA) {
    if (resultCode == RESULT_CANCELED ) {
      // User canceled
    } else if (resultCode == RESULT_OK) {
      BtConnection conn = (BtConnection)data.getParcelableExtra(BtAPI.EXTRA_BT_CONNECTION);
    conn.setListener(this);
    }
  }
}
\end{verbatim}
\normalsize

\subsection{The {\tt getIntent()} method}
\label{getIntent}
The \verb+BtAPI+ provides the two static methods \verb+getIntent()+ used to generate an \verb+Intent+ to launch a given identity provider:

\footnotesize
\begin{verbatim}
public static Intent getIntent(Context context, String idProvider);
public static Intent getIntent(Context context, String idProvider, Bundle extras);
\end{verbatim}
\normalsize

Both methods need a \verb+Context+ object and an identity provider to use (see section \ref{identity-providers} for a list). Furthermore, the second method allows for passing on a bundle of information to the identity provider -- which is used e.g. for providing the Bump\texttrademark{} with an API key.

\subsection{Identity providers}
\label{identity-providers}
An identity provider is a way of pairing two bluetooth devices and exchanging their identities in order to make a bluetooth connection. As of now, three identity providers exist:

\paragraph{{\tt BtAPI.IDENTITY\_PROVIDER\_BUMP}}
Use the Bump\texttrademark{} service to physically bump two phones together and exchange connection information\footnote{The DustyTuba project is in no way affiliated with Bump\texttrademark, we are merely using their service to obtain a bluetooth connection}.

The Bump\texttrademark{} service requires an API key, which can be obtained from their website\footnote{\url{http://bu.mp}}. This key must be provided to the identity provider as follows:

\footnotesize
\begin{verbatim}
Bundle b = new Bundle();
b.putString(BumpAPI.EXTRA_API_KEY, BUMP_API_KEY);
Intent i = BtAPI.getIntent(this, BtAPI.IDENTITY_PROVIDER_BUMP, b);
startActivityForResult(i, REQUEST_DUSTYTUBA);
\end{verbatim}
\normalsize

\paragraph{{\tt BtAPI.IDENTITY\_PROVIDER\_FAKE}}
The fake identity provider simply returns the MAC address you supply to it. This enables you to use a MAC address obtained through other means. As with the Bump\texttrademark{} identity provider, we supply the MAC address through a \verb+Bundle+:

\footnotesize
\begin{verbatim}
Bundle b = new Bundle();
b.putString(BtAPI.EXTRA_IP_MAC, "00:00:00:00:00:00");
Intent i = BtAPI.getIntent(this, BtAPI.IDENTITY_PROVIDER_FAKE, b);
startActivityForResult(i, REQUEST_DUSTYTUBA);
\end{verbatim}
\normalsize

\paragraph{{\tt BtAPI.IDENTITY\_PROVIDER\_MANUAL}}
is an identity provider allowing a user to manually enter a MAC address through a dialog. The manual identity provider is started as shown in section \ref{stepthree}. An optional default MAC address can be provided with the \verb+Bundle+ exactly as the fake MAC address was provided in the fake identity provider.

%%%%%%%%%%%%%%%%%%%%%%%%%%%%%%%%%%%%%%%%%%
%%%%%%%%%%%%%%%%%%%%%%%%%%%%%%%%%%%%%%%%%%
%%%%%%%%%%%%%%%%%%%%%%%%%%%%%%%%%%%%%%%%%%

\clearpage
\appendix

\section{Sample manifest}
\label{sample-manifest}

\footnotesize
\begin{verbatim}
<?xml version="1.0" encoding="utf-8"?>
<manifest xmlns:android="http://schemas.android.com/apk/res/android"
  package="dk.hotmovinglobster.dustytuba.apitest"
  android:versionCode="1"
  android:versionName="1.0">
  <uses-sdk android:minSdkVersion="4" />

  <activity android:name="dk.hotmovinglobster.dustytuba.bt.BluetoothConnector"></activity>
  <application android:icon="@drawable/icon" android:label="@string/app_name">
    <activity android:name=".MainActivity"
      android:label="@string/app_name">
    <intent-filter>
      <action android:name="android.intent.action.MAIN" />
      <category android:name="android.intent.category.LAUNCHER" />
    </intent-filter>
  </activity>
  <activity android:name="dk.hotmovinglobster.dustytuba.id.GenericIPActivity" />
  <activity android:name="dk.hotmovinglobster.dustytuba.id.FakeIPActivity" />
  <activity android:name="dk.hotmovinglobster.dustytuba.id.ManualIPActivity" />
  <activity android:name="dk.hotmovinglobster.dustytuba.id.BumpIPActivity"
    android:configChanges="keyboardHidden|orientation"/>
  <activity android:name="com.bumptech.bumpapi.BumpAPI"
    android:configChanges="keyboardHidden|orientation"
    android:theme="@style/BumpDialog" />
  <activity android:name="com.bumptech.bumpapi.EditTextActivity"
    android:configChanges="keyboardHidden|orientation"
   android:theme="@style/BumpDialog" />
  </application>
  
  <uses-permission android:name="android.permission.BLUETOOTH" />
  <uses-permission android:name="android.permission.BLUETOOTH_ADMIN" />  
  <uses-permission android:name="android.permission.ACCESS_FINE_LOCATION" />
  <uses-permission android:name="android.permission.ACCESS_COARSE_LOCATION" />
  <uses-permission android:name="android.permission.INTERNET" />
  <uses-permission android:name="android.permission.VIBRATE" />
  <uses-permission android:name="android.permission.READ_PHONE_STATE" />
</manifest>
\end{verbatim}
\normalsize

\end{document}
